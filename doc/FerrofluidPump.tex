%%%%%%%%%%%%%%%%%%%%%%%%%%%%%%%%%%%%%%%%%%%%%%%%%%%%%%%%%%%%%%%%%%%%%%%%
%    INSTITUTE OF PHYSICS PUBLISHING                                   %
%                                                                      %
%   `Preparing an article for publication in an Institute of Physics   %
%    Publishing journal using LaTeX'                                   %
%                                                                      %
%    LaTeX source code `ioplau2e.tex' used to generate `author         %
%    guidelines', the documentation explaining and demonstrating use   %
%    of the Institute of Physics Publishing LaTeX preprint files       %
%    `iopart.cls, iopart12.clo and iopart10.clo'.                      %
%                                                                      %
%    `ioplau2e.tex' itself uses LaTeX with `iopart.cls'                %
%                                                                      %
%%%%%%%%%%%%%%%%%%%%%%%%%%%%%%%%%%
%
%
% First we have a character check
%
% ! exclamation mark    " double quote
% # hash                ` opening quote (grave)
% & ampersand           ' closing quote (acute)
% $ dollar              % percent
% ( open parenthesis    ) close paren.
% - hyphen              = equals sign
% | vertical bar        ~ tilde
% @ at sign             _ underscore
% { open curly brace    } close curly
% [ open square         ] close square bracket
% + plus sign           ; semi-colon
% * asterisk            : colon
% < open angle bracket  > close angle
% , comma               . full stop
% ? question mark       / forward slash
% \ backslash           ^ circumflex
%
% ABCDEFGHIJKLMNOPQRSTUVWXYZ
% abcdefghijklmnopqrstuvwxyz
% 1234567890
%
%%%%%%%%%%%%%%%%%%%%%%%%%%%%%%%%%%%%%%%%%%%%%%%%%%%%%%%%%%%%%%%%%%%
%
\documentclass[12pt]{article}
\newcommand{\gguide}{{\it Preparing graphics for IOP Publishing journals}}
%Uncomment next line if AMS fonts required
%\usepackage{iopams}
\usepackage{graphicx}
\usepackage{hyperref}   % to set up hyperlinks
\hypersetup{
	colorlinks=true,
	linkcolor=blue,
	citecolor=blue,
	urlcolor=blue,
}

\begin{document}

%%\note{A novel passive
%%ferrofluid check (one-way) valve}

\title{A Novel Passive Ferrofluid One-way (Check) Valve}


%%% first author
\author{Robert L. Read}

\begin{abstract}

  Currently this is very preliminary. This is an attempt to design a pump that can efficiently pump ferrofluid with no moving parts
  except the ferrofluid and two blobs of an immiscible, incompressible fluid, such as water.
  Eliminating moving parts allows miniaturization and potentially fabrication as a chip.
  A miniature pump could be used for cooling a chip and for ``lab on a chip'' applications.
  The proposed design uses two permanent magnets and seven controllable electromagnets.
  The design was developed from considering the minimal topological connection of fixed chambers.
  It has four distinct phases, each phase having a distinct magnetic configuration.
\end{abstract}

%
% Uncomment for keywords
\vspace{2pc}
\noindent{\it Keywords}: ferrofluid pump

% Uncomment if a separate title page is required
%\maketitle
%
% For two-column output uncomment the next line and choose [10pt] rather than [12pt] in the \documentclass declaration
%\ioptwocol
%


%%%%%%%%%%%%%%%%%%%%%%%%%%%%%%%%%%%%%%%%%%%%%%%%%%%%%%%%%%%%%%%%%%%%%%
\section{Introduction}

Ferrofluid can be manipulated by electronically controlled magnetic
fields to exert force on fluids\cite{torres2014recent,kole2021engineering,ozbey2015modeling}.
This makes it possible to build pneumatic or hydraulic
devices, perhaps on very small scales,
such as a single chip\cite{yamahata2003ferrofluid,hatch2001ferrofluidic}, to
miniaturize fluid handling.
This has been proposed for biomedical purposes\cite{michelson2019novel}
that would use water or body fluids,
although this paper reports only on experiments done with air.
Miniature pumps and valves could be used to make a “lab on a chip” (LOC) or
even to heat or cool different chip areas.

%%%%%%%%%%%%%%%%%%%%%%%%%%%%%%%%%%%%%%%%%%%%%%%%%%%%%%%%%%%%%%%%%%%%%%
\section{Related Research}

A number of papers report on ferrofluid pumps, focusing in particular
on micropump and lab-on-a-chip applications\cite{ozbey2015modeling,hsu2018biocompatible}.
Many of these papers use
a version of mechanical valve not based on passive
ferrofluid, even though they move a ferrofluid bolus
with a magnetic field.
For example,
a corrugated silicone micro valve\cite{yamahata2003ferrofluid,yamahata2005plastic}
has been reported.
Other researchers use active valves, which require synchronization with
the ferrofluid plug to form a pump,
such as \cite{menz2000fluidic}, which
describes an active {\em T-Valve} with a moving ferrofluid plug, and
\cite{ando2009ferrofluidic} describes a complete fluid pump with valves
that use
active control of a ferrofluid bolus.
At least two additional kinds of active valves, a {\em well valve} and
{\em Y-valve}, have
been described\cite{hartshorne2004ferrofluid}.
Active control is possible because the
action of the plunger or bolus may be synchronized with the opening and closing
of the valves.
Nonetheless a passive valve would be simpler and less
expensive, and would not require knowledge of the timing of the
plunger.

An interesting functional micropump in which the
moving ferrofluid bolus merges with a fixed ferrofluid valve and then
separates on each pumping cycle has been described\cite{hatch2001ferrofluidic}.

An interesting devices induces a flow directly in ferrofluid
with no moving parts,
presumable based on the rotation of clumps of ferrofluid
\cite{mao2011direct}.

\section{The Idea}

\section{Conclusions}

\section{TODO}

Study this: http://citeseerx.ist.psu.edu/viewdoc/download?doi=10.1.1.719.4343&rep=rep1&type=pdf

\section{Acknowledgements}

This paper was an outgrowth the the Passive Ferrrofluid Check Valve (PFCV) \cite{stuckeynovel}
reported by Veronica Stuckey and Robert L. Read. Veronica Stuckery 3D printed
some of the apparatus.

\section*{References}

\bibliographystyle{unsrt}
% Here's where you specify the bibliography database file.
% The full file name of the bibliography database for this
% article is asme2e.bib. The name for your database is up
% to you.
\bibliography{ffcv}

\end{document}
